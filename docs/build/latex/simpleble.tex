%% Generated by Sphinx.
\def\sphinxdocclass{report}
\documentclass[letterpaper,10pt,english]{sphinxmanual}
\ifdefined\pdfpxdimen
   \let\sphinxpxdimen\pdfpxdimen\else\newdimen\sphinxpxdimen
\fi \sphinxpxdimen=.75bp\relax

\usepackage[utf8]{inputenc}
\ifdefined\DeclareUnicodeCharacter
 \ifdefined\DeclareUnicodeCharacterAsOptional
  \DeclareUnicodeCharacter{"00A0}{\nobreakspace}
  \DeclareUnicodeCharacter{"2500}{\sphinxunichar{2500}}
  \DeclareUnicodeCharacter{"2502}{\sphinxunichar{2502}}
  \DeclareUnicodeCharacter{"2514}{\sphinxunichar{2514}}
  \DeclareUnicodeCharacter{"251C}{\sphinxunichar{251C}}
  \DeclareUnicodeCharacter{"2572}{\textbackslash}
 \else
  \DeclareUnicodeCharacter{00A0}{\nobreakspace}
  \DeclareUnicodeCharacter{2500}{\sphinxunichar{2500}}
  \DeclareUnicodeCharacter{2502}{\sphinxunichar{2502}}
  \DeclareUnicodeCharacter{2514}{\sphinxunichar{2514}}
  \DeclareUnicodeCharacter{251C}{\sphinxunichar{251C}}
  \DeclareUnicodeCharacter{2572}{\textbackslash}
 \fi
\fi
\usepackage{cmap}
\usepackage[T1]{fontenc}
\usepackage{amsmath,amssymb,amstext}
\usepackage{babel}
\usepackage{times}
\usepackage[Bjarne]{fncychap}
\usepackage{sphinx}

\usepackage{geometry}

% Include hyperref last.
\usepackage{hyperref}
% Fix anchor placement for figures with captions.
\usepackage{hypcap}% it must be loaded after hyperref.
% Set up styles of URL: it should be placed after hyperref.
\urlstyle{same}
\addto\captionsenglish{\renewcommand{\contentsname}{Contents}}

\addto\captionsenglish{\renewcommand{\figurename}{Fig.}}
\addto\captionsenglish{\renewcommand{\tablename}{Table}}
\addto\captionsenglish{\renewcommand{\literalblockname}{Listing}}

\addto\captionsenglish{\renewcommand{\literalblockcontinuedname}{continued from previous page}}
\addto\captionsenglish{\renewcommand{\literalblockcontinuesname}{continues on next page}}

\addto\extrasenglish{\def\pageautorefname{page}}

\setcounter{tocdepth}{1}



\title{PandaBot Documentation}
\date{Mar 10, 2018}
\release{0.0.1}
\author{Lyudmil Vladimirov}
\newcommand{\sphinxlogo}{\vbox{}}
\renewcommand{\releasename}{Release}
\makeindex

\begin{document}

\maketitle
\sphinxtableofcontents
\phantomsection\label{\detokenize{index::doc}}



\chapter{Introduction}
\label{\detokenize{intro:welcome-to-PandaBot-s-documentation}}\label{\detokenize{intro:introduction}}\label{\detokenize{intro::doc}}
\sphinxcode{\sphinxupquote{PandaBot}} is a high-level OO Python package which aims to provide an easy and intuitive way of interacting with nearby Bluetooth Low Energy (BLE) devices (GATT servers). In essence, this package is an extension of the \sphinxcode{\sphinxupquote{bluepy}} package created by Ian Harvey (see \sphinxhref{https://github.com/IanHarvey/bluepy/}{here})

The aim here was to define a single object which would allow users to perform the various operations performed by the \sphinxcode{\sphinxupquote{bluepy.btle.Peripheral}}, \sphinxcode{\sphinxupquote{bluepy.btle.Scanner}}, \sphinxcode{\sphinxupquote{bluepy.btle.Service}} and \sphinxcode{\sphinxupquote{bluepy.btle.Characteristic}} classes of \sphinxcode{\sphinxupquote{bluepy}}, from one central place. This functionality is facilitated by the \sphinxcode{\sphinxupquote{PandaBot.PandaBotClient}} and \sphinxcode{\sphinxupquote{PandaBot.PandaBotDevice}} classes, where the latter is an extention/subclass of \sphinxcode{\sphinxupquote{bluepy.btle.Peripheral}}, combined with properties of \sphinxcode{\sphinxupquote{bluepy.btle.ScanEntry}}.

The current implementation has been developed in Python 3 and tested on a Raspberry Pi Zero W, running Raspbian 9 (stretch), but should work with Python 2.7+ (maybe with minor modifications in terms of printing and error handling) and most Debian based OSs.


\section{Motivation}
\label{\detokenize{intro:motivation}}
As a newbie experimenter/hobbyist in the field of IoT using BLE communications, I found it pretty hard to identify a Python package which would enable one to use a Raspberry Pi (Zero W inthis case) to swiftly scan, connect to and read/write from/to a nearby BLE device (GATT server).

This package is intended to provide a quick, as well as (hopefully) easy to undestand, way of getting a simple BLE GATT client up and running, for all those out there, who, like myself, are hands-on learners and are eager to get their hands dirty from early on.


\section{Limitations}
\label{\detokenize{intro:limitations}}\begin{itemize}
\item {} 
As my main use-case scenario was to simply connect two devices, the current version of {\hyperref[\detokenize{PandaBot:PandaBot.PandaBotClient}]{\sphinxcrossref{\sphinxcode{\sphinxupquote{PandaBot.PandaBotClient}}}}} has been designed and implemented with this use-case in mind. As such, if you are looking for a package to allow you to connect to multiple devices, then know that off-the-self this package DOES NOT allow you to do so. However, implementing such a feature is an easily achievable task, which has been planned for sometime in the near future and if there proves to be interest on the project, I would be happy to speed up the process.

\item {} 
Only Read and Write operations are currently supported, but I am planning on adding Notifications soon.

\item {} 
Although the interfacing operations of the \sphinxcode{\sphinxupquote{bluepy.btle.Service}} and \sphinxcode{\sphinxupquote{bluepy.btle.Peripheral}} classes have been brought forward to the {\hyperref[\detokenize{PandaBot:PandaBot.PandaBotClient}]{\sphinxcrossref{\sphinxcode{\sphinxupquote{PandaBot.PandaBotClient}}}}} class, the same has not been done for the \sphinxcode{\sphinxupquote{bluepy.btle.Descriptor}}, meaning that the {\hyperref[\detokenize{PandaBot:PandaBot.PandaBotClient}]{\sphinxcrossref{\sphinxcode{\sphinxupquote{PandaBot.PandaBotClient}}}}} cannot be used to directly access the Descriptors. This can however be done easily by obtaining a handle of a {\hyperref[\detokenize{PandaBot:PandaBot.PandaBotDevice}]{\sphinxcrossref{\sphinxcode{\sphinxupquote{PandaBot.PandaBotDevice}}}}} object and calling the superclass \sphinxcode{\sphinxupquote{bluepy.btle.Peripheral.getDescriptors()}} method.

\end{itemize}


\chapter{Documentation}
\label{\detokenize{PandaBot:documentation}}\label{\detokenize{PandaBot::doc}}

\section{The \sphinxstyleliteralintitle{\sphinxupquote{PandaBotClient}} class}
\label{\detokenize{PandaBot:the-PandaBotclient-class}}\index{PandaBotClient (class in PandaBot)}

\begin{fulllineitems}
\phantomsection\label{\detokenize{PandaBot:PandaBot.PandaBotClient}}\pysiglinewithargsret{\sphinxbfcode{\sphinxupquote{class }}\sphinxcode{\sphinxupquote{PandaBot.}}\sphinxbfcode{\sphinxupquote{PandaBotClient}}}{\emph{iface=0}, \emph{scanCallback=None}, \emph{notificationCallback=None}}{}
Bases: \sphinxcode{\sphinxupquote{object}}

This is a class implementation of a simple BLE client.
\begin{quote}\begin{description}
\item[{Parameters}] \leavevmode\begin{itemize}
\item {} 
\sphinxstyleliteralstrong{\sphinxupquote{iface}} (\sphinxstyleliteralemphasis{\sphinxupquote{int}}\sphinxstyleliteralemphasis{\sphinxupquote{, }}\sphinxstyleliteralemphasis{\sphinxupquote{optional}}) \textendash{} The Bluetooth interface on which to make the connection. On Linux, 0 means \sphinxtitleref{/dev/hci0}, 1 means \sphinxtitleref{/dev/hci1} and so on., defaults to 0

\item {} 
\sphinxstyleliteralstrong{\sphinxupquote{scanCallback}} (\sphinxstyleliteralemphasis{\sphinxupquote{function}}\sphinxstyleliteralemphasis{\sphinxupquote{, }}\sphinxstyleliteralemphasis{\sphinxupquote{optional}}) \textendash{} A function handle of the form \sphinxcode{\sphinxupquote{callback(client, device, isNewDevice, isNewData)}}, where \sphinxcode{\sphinxupquote{client}} is a handle to the {\hyperref[\detokenize{PandaBot:PandaBot.PandaBotClient}]{\sphinxcrossref{\sphinxcode{\sphinxupquote{PandaBot.PandaBotClient}}}}} that invoked the callback and \sphinxcode{\sphinxupquote{device}} is the detected {\hyperref[\detokenize{PandaBot:PandaBot.PandaBotDevice}]{\sphinxcrossref{\sphinxcode{\sphinxupquote{PandaBot.PandaBotDevice}}}}} object. \sphinxcode{\sphinxupquote{isNewDev}} is \sphinxtitleref{True} if the device (as identified by its MAC address) has not been seen before by the scanner, and \sphinxtitleref{False} otherwise. \sphinxcode{\sphinxupquote{isNewData}} is \sphinxtitleref{True} if new or updated advertising data is available, defaults to None

\item {} 
\sphinxstyleliteralstrong{\sphinxupquote{notificationCallback}} (\sphinxstyleliteralemphasis{\sphinxupquote{function}}\sphinxstyleliteralemphasis{\sphinxupquote{, }}\sphinxstyleliteralemphasis{\sphinxupquote{optional}}) \textendash{} A function handle of the form \sphinxcode{\sphinxupquote{callback(client, characteristic, data)}}, where \sphinxcode{\sphinxupquote{client}} is a handle to the {\hyperref[\detokenize{PandaBot:PandaBot.PandaBotClient}]{\sphinxcrossref{\sphinxcode{\sphinxupquote{PandaBot.PandaBotClient}}}}} that invoked the callback, \sphinxcode{\sphinxupquote{characteristic}} is the notified \sphinxcode{\sphinxupquote{bluepy.blte.Characteristic}} object and data is a \sphinxtitleref{bytearray} containing the updated value. Defaults to None

\end{itemize}

\end{description}\end{quote}
\index{connect() (PandaBot.PandaBotClient method)}

\begin{fulllineitems}
\phantomsection\label{\detokenize{PandaBot:PandaBot.PandaBotClient.connect}}\pysiglinewithargsret{\sphinxbfcode{\sphinxupquote{connect}}}{\emph{device}}{}
Attempts to connect client to a given {\hyperref[\detokenize{PandaBot:PandaBot.PandaBotDevice}]{\sphinxcrossref{\sphinxcode{\sphinxupquote{PandaBot.PandaBotDevice}}}}} object and returns a bool indication of the result.
\begin{quote}\begin{description}
\item[{Parameters}] \leavevmode
\sphinxstyleliteralstrong{\sphinxupquote{device}} ({\hyperref[\detokenize{PandaBot:PandaBot.PandaBotDevice}]{\sphinxcrossref{\sphinxstyleliteralemphasis{\sphinxupquote{PandaBotDevice}}}}}) \textendash{} An instance of the device to which we want to connect. Normally acquired by calling {\hyperref[\detokenize{PandaBot:PandaBot.PandaBotClient.scan}]{\sphinxcrossref{\sphinxcode{\sphinxupquote{PandaBot.PandaBotClient.scan()}}}}} or {\hyperref[\detokenize{PandaBot:PandaBot.PandaBotClient.searchDevice}]{\sphinxcrossref{\sphinxcode{\sphinxupquote{PandaBot.PandaBotClient.searchDevice()}}}}}

\item[{Returns}] \leavevmode
\sphinxtitleref{True} if connection was successful, \sphinxtitleref{False} otherwise

\item[{Return type}] \leavevmode
bool

\end{description}\end{quote}

\end{fulllineitems}

\index{disconnect() (PandaBot.PandaBotClient method)}

\begin{fulllineitems}
\phantomsection\label{\detokenize{PandaBot:PandaBot.PandaBotClient.disconnect}}\pysiglinewithargsret{\sphinxbfcode{\sphinxupquote{disconnect}}}{}{}
Drops existing connection. 
Note that the current version of the project assumes that the client can be connected to at most one device at a time.

\end{fulllineitems}

\index{getCharacteristics() (PandaBot.PandaBotClient method)}

\begin{fulllineitems}
\phantomsection\label{\detokenize{PandaBot:PandaBot.PandaBotClient.getCharacteristics}}\pysiglinewithargsret{\sphinxbfcode{\sphinxupquote{getCharacteristics}}}{\emph{startHnd=1}, \emph{endHnd=65535}, \emph{uuids=None}}{}
Returns a list containing \sphinxcode{\sphinxupquote{bluepy.btle.Characteristic}} objects for the peripheral. If no arguments are given, will return all characteristics. If startHnd and/or endHnd are given, the list is restricted to characteristics whose handles are within the given range.
\begin{quote}\begin{description}
\item[{Parameters}] \leavevmode\begin{itemize}
\item {} 
\sphinxstyleliteralstrong{\sphinxupquote{startHnd}} (\sphinxstyleliteralemphasis{\sphinxupquote{int}}\sphinxstyleliteralemphasis{\sphinxupquote{, }}\sphinxstyleliteralemphasis{\sphinxupquote{optional}}) \textendash{} Start index, defaults to 1

\item {} 
\sphinxstyleliteralstrong{\sphinxupquote{endHnd}} (\sphinxstyleliteralemphasis{\sphinxupquote{int}}\sphinxstyleliteralemphasis{\sphinxupquote{, }}\sphinxstyleliteralemphasis{\sphinxupquote{optional}}) \textendash{} End index, defaults to 0xFFFF

\item {} 
\sphinxstyleliteralstrong{\sphinxupquote{uuids}} (\sphinxstyleliteralemphasis{\sphinxupquote{list}}\sphinxstyleliteralemphasis{\sphinxupquote{, }}\sphinxstyleliteralemphasis{\sphinxupquote{optional}}) \textendash{} a list of UUID strings, defaults to None

\end{itemize}

\item[{Returns}] \leavevmode
List of returned \sphinxcode{\sphinxupquote{bluepy.btle.Characteristic}} objects

\item[{Return type}] \leavevmode
list

\end{description}\end{quote}

\end{fulllineitems}

\index{isConnected() (PandaBot.PandaBotClient method)}

\begin{fulllineitems}
\phantomsection\label{\detokenize{PandaBot:PandaBot.PandaBotClient.isConnected}}\pysiglinewithargsret{\sphinxbfcode{\sphinxupquote{isConnected}}}{}{}
Check to see if client is connected to a device
\begin{quote}\begin{description}
\item[{Returns}] \leavevmode
\sphinxtitleref{True} if connected, \sphinxtitleref{False} otherwise

\item[{Return type}] \leavevmode
bool

\end{description}\end{quote}

\end{fulllineitems}

\index{printFoundDevices() (PandaBot.PandaBotClient method)}

\begin{fulllineitems}
\phantomsection\label{\detokenize{PandaBot:PandaBot.PandaBotClient.printFoundDevices}}\pysiglinewithargsret{\sphinxbfcode{\sphinxupquote{printFoundDevices}}}{}{}
Print all devices discovered during the last scan. Should only be called after a {\hyperref[\detokenize{PandaBot:PandaBot.PandaBotClient.scan}]{\sphinxcrossref{\sphinxcode{\sphinxupquote{PandaBot.PandaBotClient.scan()}}}}} has been called first.

\end{fulllineitems}

\index{readCharacteristic() (PandaBot.PandaBotClient method)}

\begin{fulllineitems}
\phantomsection\label{\detokenize{PandaBot:PandaBot.PandaBotClient.readCharacteristic}}\pysiglinewithargsret{\sphinxbfcode{\sphinxupquote{readCharacteristic}}}{\emph{characteristic=None}, \emph{uuid=None}}{}
Reads the current value of the characteristic identified by either a \sphinxcode{\sphinxupquote{bluepy.btle.Characteristic}} object \sphinxcode{\sphinxupquote{characteristic}}, or a UUID string \sphinxcode{\sphinxupquote{uuid}}. If both are provided, then the characteristic will be read on the basis of the \sphinxcode{\sphinxupquote{characteristic}} object. A \sphinxcode{\sphinxupquote{bluepy.btle.BTLEException.GATT\_ERROR}} is raised if no inputs are specified or the requested characteristic was not found.
\begin{quote}\begin{description}
\item[{Parameters}] \leavevmode\begin{itemize}
\item {} 
\sphinxstyleliteralstrong{\sphinxupquote{characteristic}} (\sphinxcode{\sphinxupquote{bluepy.btle.Characteristic}}, optional) \textendash{} A \sphinxcode{\sphinxupquote{bluepy.btle.Characteristic}} object, defaults to None

\item {} 
\sphinxstyleliteralstrong{\sphinxupquote{uuid}} (\sphinxstyleliteralemphasis{\sphinxupquote{string}}\sphinxstyleliteralemphasis{\sphinxupquote{, }}\sphinxstyleliteralemphasis{\sphinxupquote{optional}}) \textendash{} A given UUID string, defaults to None

\end{itemize}

\item[{Raises}] \leavevmode
\sphinxcode{\sphinxupquote{bluepy.btle.BTLEException.GATT\_ERROR}}: If no inputs are specified or the requested characteristic was not found.

\item[{Returns}] \leavevmode
The value read from the characteristic

\item[{Return type}] \leavevmode
bytearray

\end{description}\end{quote}

\end{fulllineitems}

\index{scan() (PandaBot.PandaBotClient method)}

\begin{fulllineitems}
\phantomsection\label{\detokenize{PandaBot:PandaBot.PandaBotClient.scan}}\pysiglinewithargsret{\sphinxbfcode{\sphinxupquote{scan}}}{\emph{timeout=10.0}}{}
Scans for and returns detected nearby devices
\begin{quote}\begin{description}
\item[{Parameters}] \leavevmode
\sphinxstyleliteralstrong{\sphinxupquote{timeout}} (\sphinxstyleliteralemphasis{\sphinxupquote{float}}\sphinxstyleliteralemphasis{\sphinxupquote{, }}\sphinxstyleliteralemphasis{\sphinxupquote{optional}}) \textendash{} Specify how long (in seconds) the scan should last, defaults to 10.0

\item[{Returns}] \leavevmode
List of {\hyperref[\detokenize{PandaBot:PandaBot.PandaBotDevice}]{\sphinxcrossref{\sphinxcode{\sphinxupquote{PandaBot.PandaBotDevice}}}}} objects

\item[{Return type}] \leavevmode
list

\end{description}\end{quote}

\end{fulllineitems}

\index{searchDevice() (PandaBot.PandaBotClient method)}

\begin{fulllineitems}
\phantomsection\label{\detokenize{PandaBot:PandaBot.PandaBotClient.searchDevice}}\pysiglinewithargsret{\sphinxbfcode{\sphinxupquote{searchDevice}}}{\emph{name=None}, \emph{mac=None}, \emph{timeout=10.0}}{}
Searches for and returns, given it exists, a {\hyperref[\detokenize{PandaBot:PandaBot.PandaBotDevice}]{\sphinxcrossref{\sphinxcode{\sphinxupquote{PandaBot.PandaBotDevice}}}}} device objects, based on the provided \sphinxcode{\sphinxupquote{name}} and/or \sphinxcode{\sphinxupquote{mac}} address. If both a \sphinxcode{\sphinxupquote{name}} and a \sphinxcode{\sphinxupquote{mac}} are provided, then the client will only return a device that matches both conditions.
\begin{quote}\begin{description}
\item[{Parameters}] \leavevmode\begin{itemize}
\item {} 
\sphinxstyleliteralstrong{\sphinxupquote{name}} (\sphinxstyleliteralemphasis{\sphinxupquote{str}}\sphinxstyleliteralemphasis{\sphinxupquote{, }}\sphinxstyleliteralemphasis{\sphinxupquote{optional}}) \textendash{} The “Complete Local Name” Generic Access Attribute (GATT) of the device, defaults to None

\item {} 
\sphinxstyleliteralstrong{\sphinxupquote{mac}} (\sphinxstyleliteralemphasis{\sphinxupquote{str}}\sphinxstyleliteralemphasis{\sphinxupquote{, }}\sphinxstyleliteralemphasis{\sphinxupquote{optional}}) \textendash{} The MAC address of the device, defaults to None

\item {} 
\sphinxstyleliteralstrong{\sphinxupquote{timeout}} (\sphinxstyleliteralemphasis{\sphinxupquote{float}}\sphinxstyleliteralemphasis{\sphinxupquote{, }}\sphinxstyleliteralemphasis{\sphinxupquote{optional}}) \textendash{} Specify how long (in seconds) the scan should last, defaults to 10.0. Internally, it serves as an input to the invoked {\hyperref[\detokenize{PandaBot:PandaBot.PandaBotClient.scan}]{\sphinxcrossref{\sphinxcode{\sphinxupquote{PandaBot.PandaBotClient.scan()}}}}} method.

\end{itemize}

\item[{Raises}] \leavevmode
\sphinxstyleliteralstrong{\sphinxupquote{AssertionError}} \textendash{} If neither a \sphinxcode{\sphinxupquote{name}} nor a \sphinxcode{\sphinxupquote{mac}} inputs have been provided

\item[{Returns}] \leavevmode
A {\hyperref[\detokenize{PandaBot:PandaBot.PandaBotDevice}]{\sphinxcrossref{\sphinxcode{\sphinxupquote{PandaBot.PandaBotDevice}}}}} object if search was succesfull, None otherwise

\item[{Return type}] \leavevmode
{\hyperref[\detokenize{PandaBot:PandaBot.PandaBotDevice}]{\sphinxcrossref{\sphinxcode{\sphinxupquote{PandaBot.PandaBotDevice}}}}} \textbar{} None

\end{description}\end{quote}

\end{fulllineitems}

\index{setNotificationCallback() (PandaBot.PandaBotClient method)}

\begin{fulllineitems}
\phantomsection\label{\detokenize{PandaBot:PandaBot.PandaBotClient.setNotificationCallback}}\pysiglinewithargsret{\sphinxbfcode{\sphinxupquote{setNotificationCallback}}}{\emph{callback}}{}
Set the callback function to be executed when a device sends a notification to the client.
\begin{quote}\begin{description}
\item[{Parameters}] \leavevmode
\sphinxstyleliteralstrong{\sphinxupquote{callback}} (\sphinxstyleliteralemphasis{\sphinxupquote{function}}\sphinxstyleliteralemphasis{\sphinxupquote{, }}\sphinxstyleliteralemphasis{\sphinxupquote{optional}}) \textendash{} A function handle of the form \sphinxcode{\sphinxupquote{callback(client, characteristic, data)}}, where \sphinxcode{\sphinxupquote{client}} is a handle to the {\hyperref[\detokenize{PandaBot:PandaBot.PandaBotClient}]{\sphinxcrossref{\sphinxcode{\sphinxupquote{PandaBot.PandaBotClient}}}}} that invoked the callback, \sphinxcode{\sphinxupquote{characteristic}} is the notified \sphinxcode{\sphinxupquote{bluepy.blte.Characteristic}} object and data is a \sphinxtitleref{bytearray} containing the updated value. Defaults to None

\end{description}\end{quote}

\end{fulllineitems}

\index{setScanCallback() (PandaBot.PandaBotClient method)}

\begin{fulllineitems}
\phantomsection\label{\detokenize{PandaBot:PandaBot.PandaBotClient.setScanCallback}}\pysiglinewithargsret{\sphinxbfcode{\sphinxupquote{setScanCallback}}}{\emph{callback}}{}
Set the callback function to be executed when a device is detected by the client.
\begin{quote}\begin{description}
\item[{Parameters}] \leavevmode
\sphinxstyleliteralstrong{\sphinxupquote{callback}} (\sphinxstyleliteralemphasis{\sphinxupquote{function}}) \textendash{} A function handle of the form \sphinxcode{\sphinxupquote{callback(client, device, isNewDevice, isNewData)}}, where \sphinxcode{\sphinxupquote{client}} is a handle to the {\hyperref[\detokenize{PandaBot:PandaBot.PandaBotClient}]{\sphinxcrossref{\sphinxcode{\sphinxupquote{PandaBot.PandaBotClient}}}}} that invoked the callback and \sphinxcode{\sphinxupquote{device}} is the detected {\hyperref[\detokenize{PandaBot:PandaBot.PandaBotDevice}]{\sphinxcrossref{\sphinxcode{\sphinxupquote{PandaBot.PandaBotDevice}}}}} object. \sphinxcode{\sphinxupquote{isNewDev}} is \sphinxtitleref{True} if the device (as identified by its MAC address) has not been seen before by the scanner, and \sphinxtitleref{False} otherwise. \sphinxcode{\sphinxupquote{isNewData}} is \sphinxtitleref{True} if new or updated advertising data is available.

\end{description}\end{quote}

\end{fulllineitems}

\index{writeCharacteristic() (PandaBot.PandaBotClient method)}

\begin{fulllineitems}
\phantomsection\label{\detokenize{PandaBot:PandaBot.PandaBotClient.writeCharacteristic}}\pysiglinewithargsret{\sphinxbfcode{\sphinxupquote{writeCharacteristic}}}{\emph{val}, \emph{characteristic=None}, \emph{uuid=None}, \emph{withResponse=False}}{}
Writes the data val (of type str on Python 2.x, byte on 3.x) to the characteristic identified by either a \sphinxcode{\sphinxupquote{bluepy.btle.Characteristic}} object \sphinxcode{\sphinxupquote{characteristic}}, or a UUID string \sphinxcode{\sphinxupquote{uuid}}. If both are provided, then the characteristic will be read on the basis of the \sphinxcode{\sphinxupquote{characteristic}} object. A \sphinxcode{\sphinxupquote{bluepy.btle.BTLEException.GATT\_ERROR}} is raised if no inputs are specified or the requested characteristic was not found. If \sphinxcode{\sphinxupquote{withResponse}} is \sphinxtitleref{True}, the client will await confirmation that the write was successful from the device.
\begin{quote}\begin{description}
\item[{Parameters}] \leavevmode\begin{itemize}
\item {} 
\sphinxstyleliteralstrong{\sphinxupquote{val}} (\sphinxstyleliteralemphasis{\sphinxupquote{str on Python 2.x}}\sphinxstyleliteralemphasis{\sphinxupquote{, }}\sphinxstyleliteralemphasis{\sphinxupquote{byte on 3.x}}) \textendash{} Value to be written in characteristic

\item {} 
\sphinxstyleliteralstrong{\sphinxupquote{characteristic}} (\sphinxcode{\sphinxupquote{bluepy.btle.Characteristic}}, optional) \textendash{} A \sphinxcode{\sphinxupquote{bluepy.btle.Characteristic}} object, defaults to None

\item {} 
\sphinxstyleliteralstrong{\sphinxupquote{uuid}} (\sphinxstyleliteralemphasis{\sphinxupquote{string}}\sphinxstyleliteralemphasis{\sphinxupquote{, }}\sphinxstyleliteralemphasis{\sphinxupquote{optional}}) \textendash{} A given UUID string, defaults to None

\item {} 
\sphinxstyleliteralstrong{\sphinxupquote{withResponse}} (\sphinxstyleliteralemphasis{\sphinxupquote{bool}}\sphinxstyleliteralemphasis{\sphinxupquote{, }}\sphinxstyleliteralemphasis{\sphinxupquote{optional}}) \textendash{} If \sphinxcode{\sphinxupquote{withResponse}} is \sphinxtitleref{True}, the client will await confirmation that the write was successful from the device, defaults to False

\end{itemize}

\item[{Raises}] \leavevmode
\sphinxcode{\sphinxupquote{bluepy.btle.BTLEException.GATT\_ERROR}}: If no inputs are specified or the requested characteristic was not found.

\item[{Returns}] \leavevmode
\sphinxtitleref{True} or \sphinxtitleref{False} indicating success or failure of write operation, in the case that \sphinxcode{\sphinxupquote{withResponce}} is \sphinxtitleref{True}

\item[{Return type}] \leavevmode
bool

\end{description}\end{quote}

\end{fulllineitems}


\end{fulllineitems}



\section{The \sphinxstyleliteralintitle{\sphinxupquote{PandaBotDevice}} class}
\label{\detokenize{PandaBot:the-PandaBotdevice-class}}\index{PandaBotDevice (class in PandaBot)}

\begin{fulllineitems}
\phantomsection\label{\detokenize{PandaBot:PandaBot.PandaBotDevice}}\pysiglinewithargsret{\sphinxbfcode{\sphinxupquote{class }}\sphinxcode{\sphinxupquote{PandaBot.}}\sphinxbfcode{\sphinxupquote{PandaBotDevice}}}{\emph{client}, \emph{addr=None}, \emph{addrType='public'}, \emph{iface=0}, \emph{data=None}, \emph{rssi=0}, \emph{connectable=False}, \emph{updateCount=0}}{}
Bases: \sphinxcode{\sphinxupquote{bluepy.btle.Peripheral}}

This is a conceptual class representation of a simple BLE device (GATT Server). It is essentially an extended combination of the \sphinxcode{\sphinxupquote{bluepy.btle.Peripheral}} and \sphinxcode{\sphinxupquote{bluepy.btle.ScanEntry}} classes
\begin{quote}\begin{description}
\item[{Parameters}] \leavevmode\begin{itemize}
\item {} 
\sphinxstyleliteralstrong{\sphinxupquote{client}} (class:\sphinxtitleref{PandaBot.PandaBotClient}) \textendash{} A handle to the {\hyperref[\detokenize{PandaBot:PandaBot.PandaBotClient}]{\sphinxcrossref{\sphinxcode{\sphinxupquote{PandaBot.PandaBotClient}}}}} client object that detected the device

\item {} 
\sphinxstyleliteralstrong{\sphinxupquote{addr}} (\sphinxstyleliteralemphasis{\sphinxupquote{str}}\sphinxstyleliteralemphasis{\sphinxupquote{, }}\sphinxstyleliteralemphasis{\sphinxupquote{optional}}) \textendash{} Device MAC address, defaults to None

\item {} 
\sphinxstyleliteralstrong{\sphinxupquote{addrType}} (\sphinxstyleliteralemphasis{\sphinxupquote{str}}\sphinxstyleliteralemphasis{\sphinxupquote{, }}\sphinxstyleliteralemphasis{\sphinxupquote{optional}}) \textendash{} Device address type - one of ADDR\_TYPE\_PUBLIC or ADDR\_TYPE\_RANDOM, defaults to ADDR\_TYPE\_PUBLIC

\item {} 
\sphinxstyleliteralstrong{\sphinxupquote{iface}} (\sphinxstyleliteralemphasis{\sphinxupquote{int}}\sphinxstyleliteralemphasis{\sphinxupquote{, }}\sphinxstyleliteralemphasis{\sphinxupquote{optional}}) \textendash{} Bluetooth interface number (0 = /dev/hci0) used for the connection, defaults to 0

\item {} 
\sphinxstyleliteralstrong{\sphinxupquote{data}} (\sphinxstyleliteralemphasis{\sphinxupquote{list}}\sphinxstyleliteralemphasis{\sphinxupquote{, }}\sphinxstyleliteralemphasis{\sphinxupquote{optional}}) \textendash{} A list of tuples (adtype, description, value) containing the AD type code, human-readable description and value for all available advertising data items, defaults to None

\item {} 
\sphinxstyleliteralstrong{\sphinxupquote{rssi}} (\sphinxstyleliteralemphasis{\sphinxupquote{int}}\sphinxstyleliteralemphasis{\sphinxupquote{, }}\sphinxstyleliteralemphasis{\sphinxupquote{optional}}) \textendash{} Received Signal Strength Indication for the last received broadcast from the device. This is an integer value measured in dB, where 0 dB is the maximum (theoretical) signal strength, and more negative numbers indicate a weaker signal, defaults to 0

\item {} 
\sphinxstyleliteralstrong{\sphinxupquote{connectable}} (\sphinxstyleliteralemphasis{\sphinxupquote{bool}}\sphinxstyleliteralemphasis{\sphinxupquote{, }}\sphinxstyleliteralemphasis{\sphinxupquote{optional}}) \textendash{} \sphinxtitleref{True} if the device supports connections, and \sphinxtitleref{False} otherwise (typically used for advertising ‘beacons’)., defaults to \sphinxtitleref{False}

\item {} 
\sphinxstyleliteralstrong{\sphinxupquote{updateCount}} (\sphinxstyleliteralemphasis{\sphinxupquote{int}}\sphinxstyleliteralemphasis{\sphinxupquote{, }}\sphinxstyleliteralemphasis{\sphinxupquote{optional}}) \textendash{} Integer count of the number of advertising packets received from the device so far, defaults to 0

\end{itemize}

\end{description}\end{quote}
\index{connect() (PandaBot.PandaBotDevice method)}

\begin{fulllineitems}
\phantomsection\label{\detokenize{PandaBot:PandaBot.PandaBotDevice.connect}}\pysiglinewithargsret{\sphinxbfcode{\sphinxupquote{connect}}}{}{}
Attempts to initiate a connection with the device.
\begin{quote}\begin{description}
\item[{Returns}] \leavevmode
\sphinxtitleref{True} if connection was successful, \sphinxtitleref{False} otherwise

\item[{Return type}] \leavevmode
bool

\end{description}\end{quote}

\end{fulllineitems}

\index{disconnect() (PandaBot.PandaBotDevice method)}

\begin{fulllineitems}
\phantomsection\label{\detokenize{PandaBot:PandaBot.PandaBotDevice.disconnect}}\pysiglinewithargsret{\sphinxbfcode{\sphinxupquote{disconnect}}}{}{}
Drops existing connection to device

\end{fulllineitems}

\index{getCharacteristics() (PandaBot.PandaBotDevice method)}

\begin{fulllineitems}
\phantomsection\label{\detokenize{PandaBot:PandaBot.PandaBotDevice.getCharacteristics}}\pysiglinewithargsret{\sphinxbfcode{\sphinxupquote{getCharacteristics}}}{\emph{startHnd=1}, \emph{endHnd=65535}, \emph{uuids=None}}{}
Returns a list containing \sphinxcode{\sphinxupquote{bluepy.btle.Characteristic}} objects for the peripheral. If no arguments are given, will return all characteristics. If startHnd and/or endHnd are given, the list is restricted to characteristics whose handles are within the given range.
\begin{quote}\begin{description}
\item[{Parameters}] \leavevmode\begin{itemize}
\item {} 
\sphinxstyleliteralstrong{\sphinxupquote{startHnd}} (\sphinxstyleliteralemphasis{\sphinxupquote{int}}\sphinxstyleliteralemphasis{\sphinxupquote{, }}\sphinxstyleliteralemphasis{\sphinxupquote{optional}}) \textendash{} Start index, defaults to 1

\item {} 
\sphinxstyleliteralstrong{\sphinxupquote{endHnd}} (\sphinxstyleliteralemphasis{\sphinxupquote{int}}\sphinxstyleliteralemphasis{\sphinxupquote{, }}\sphinxstyleliteralemphasis{\sphinxupquote{optional}}) \textendash{} End index, defaults to 0xFFFF

\item {} 
\sphinxstyleliteralstrong{\sphinxupquote{uuids}} (\sphinxstyleliteralemphasis{\sphinxupquote{list}}\sphinxstyleliteralemphasis{\sphinxupquote{, }}\sphinxstyleliteralemphasis{\sphinxupquote{optional}}) \textendash{} a list of UUID strings, defaults to None

\end{itemize}

\item[{Returns}] \leavevmode
List of returned \sphinxcode{\sphinxupquote{bluepy.btle.Characteristic}} objects

\item[{Return type}] \leavevmode
list

\end{description}\end{quote}

\end{fulllineitems}

\index{getServices() (PandaBot.PandaBotDevice method)}

\begin{fulllineitems}
\phantomsection\label{\detokenize{PandaBot:PandaBot.PandaBotDevice.getServices}}\pysiglinewithargsret{\sphinxbfcode{\sphinxupquote{getServices}}}{\emph{uuids=None}}{}
Returns a list of \sphinxcode{\sphinxupquote{bluepy.blte.Service}} objects representing the services offered by the device. This will perform Bluetooth service discovery if this has not already been done; otherwise it will return a cached list of services immediately..
\begin{quote}\begin{description}
\item[{Parameters}] \leavevmode
\sphinxstyleliteralstrong{\sphinxupquote{uuids}} (\sphinxstyleliteralemphasis{\sphinxupquote{list}}\sphinxstyleliteralemphasis{\sphinxupquote{, }}\sphinxstyleliteralemphasis{\sphinxupquote{optional}}) \textendash{} A list of string service UUIDs to be discovered, defaults to None

\item[{Returns}] \leavevmode
A list of the discovered \sphinxcode{\sphinxupquote{bluepy.blte.Service}} objects, which match the provided \sphinxcode{\sphinxupquote{uuids}}

\item[{Return type}] \leavevmode
list On Python 3.x, this returns a dictionary view object, not a list

\end{description}\end{quote}

\end{fulllineitems}

\index{isConnected() (PandaBot.PandaBotDevice method)}

\begin{fulllineitems}
\phantomsection\label{\detokenize{PandaBot:PandaBot.PandaBotDevice.isConnected}}\pysiglinewithargsret{\sphinxbfcode{\sphinxupquote{isConnected}}}{}{}
Checks to see if device is connected
\begin{quote}\begin{description}
\item[{Returns}] \leavevmode
\sphinxtitleref{True} if connected, \sphinxtitleref{False} otherwise

\item[{Return type}] \leavevmode
bool

\end{description}\end{quote}

\end{fulllineitems}

\index{printInfo() (PandaBot.PandaBotDevice method)}

\begin{fulllineitems}
\phantomsection\label{\detokenize{PandaBot:PandaBot.PandaBotDevice.printInfo}}\pysiglinewithargsret{\sphinxbfcode{\sphinxupquote{printInfo}}}{}{}
Print info about device

\end{fulllineitems}

\index{setNotificationCallback() (PandaBot.PandaBotDevice method)}

\begin{fulllineitems}
\phantomsection\label{\detokenize{PandaBot:PandaBot.PandaBotDevice.setNotificationCallback}}\pysiglinewithargsret{\sphinxbfcode{\sphinxupquote{setNotificationCallback}}}{\emph{callback}}{}
Set the callback function to be executed when the device sends a notification to the client.
\begin{quote}\begin{description}
\item[{Parameters}] \leavevmode
\sphinxstyleliteralstrong{\sphinxupquote{callback}} (\sphinxstyleliteralemphasis{\sphinxupquote{function}}\sphinxstyleliteralemphasis{\sphinxupquote{, }}\sphinxstyleliteralemphasis{\sphinxupquote{optional}}) \textendash{} A function handle of the form \sphinxcode{\sphinxupquote{callback(client, characteristic, data)}}, where \sphinxcode{\sphinxupquote{client}} is a handle to the {\hyperref[\detokenize{PandaBot:PandaBot.PandaBotClient}]{\sphinxcrossref{\sphinxcode{\sphinxupquote{PandaBot.PandaBotClient}}}}} that invoked the callback, \sphinxcode{\sphinxupquote{characteristic}} is the notified \sphinxcode{\sphinxupquote{bluepy.blte.Characteristic}} object and data is a \sphinxtitleref{bytearray} containing the updated value. Defaults to None

\end{description}\end{quote}

\end{fulllineitems}


\end{fulllineitems}



\section{The \sphinxstyleliteralintitle{\sphinxupquote{PandaBotScanDelegate}} class}
\label{\detokenize{PandaBot:the-PandaBotscandelegate-class}}\index{PandaBotScanDelegate (class in PandaBot)}

\begin{fulllineitems}
\phantomsection\label{\detokenize{PandaBot:PandaBot.PandaBotScanDelegate}}\pysiglinewithargsret{\sphinxbfcode{\sphinxupquote{class }}\sphinxcode{\sphinxupquote{PandaBot.}}\sphinxbfcode{\sphinxupquote{PandaBotScanDelegate}}}{\emph{callback}, \emph{client=None}}{}
Bases: \sphinxcode{\sphinxupquote{bluepy.btle.DefaultDelegate}}
\index{handleDiscovery() (PandaBot.PandaBotScanDelegate method)}

\begin{fulllineitems}
\phantomsection\label{\detokenize{PandaBot:PandaBot.PandaBotScanDelegate.handleDiscovery}}\pysiglinewithargsret{\sphinxbfcode{\sphinxupquote{handleDiscovery}}}{\emph{scanEntry}, \emph{isNewDevice}, \emph{isNewData}}{}
\end{fulllineitems}


\end{fulllineitems}



\section{The \sphinxstyleliteralintitle{\sphinxupquote{PandaBotNotificationDelegate}} class}
\label{\detokenize{PandaBot:the-PandaBotnotificationdelegate-class}}\index{PandaBotNotificationDelegate (class in PandaBot)}

\begin{fulllineitems}
\phantomsection\label{\detokenize{PandaBot:PandaBot.PandaBotNotificationDelegate}}\pysiglinewithargsret{\sphinxbfcode{\sphinxupquote{class }}\sphinxcode{\sphinxupquote{PandaBot.}}\sphinxbfcode{\sphinxupquote{PandaBotNotificationDelegate}}}{\emph{callback}, \emph{client}}{}
Bases: \sphinxcode{\sphinxupquote{bluepy.btle.DefaultDelegate}}
\index{handleNotification() (PandaBot.PandaBotNotificationDelegate method)}

\begin{fulllineitems}
\phantomsection\label{\detokenize{PandaBot:PandaBot.PandaBotNotificationDelegate.handleNotification}}\pysiglinewithargsret{\sphinxbfcode{\sphinxupquote{handleNotification}}}{\emph{characteristic}, \emph{data}}{}
\end{fulllineitems}


\end{fulllineitems}



\chapter{Example Codes}
\label{\detokenize{examples:example-codes}}\label{\detokenize{examples::doc}}

\section{Installation/Usage:}
\label{\detokenize{examples:installation-usage}}
As the package has not been published on PyPi yet, it CANNOT be install using pip.

For now, the suggested method is to put the file \sphinxtitleref{PandaBot.py} in the same directory as your source files and call \sphinxcode{\sphinxupquote{from PandaBot import PandaBotClient, PandaBotDevice}}.

\sphinxcode{\sphinxupquote{bluepy}} must also be installed and imported as shown in the example below.
For instructions about how to install, as well as the full documentation of, \sphinxcode{\sphinxupquote{bluepy}} please refer \sphinxhref{https://github.com/IanHarvey/bluepy/}{here}


\section{Search for device, connect and read characteristic}
\label{\detokenize{examples:search-for-device-connect-and-read-characteristic}}
\fvset{hllines={, ,}}%
\begin{sphinxVerbatim}[commandchars=\\\{\}]
\PYG{l+s+sd}{\PYGZdq{}\PYGZdq{}\PYGZdq{}This example demonstrates a simple BLE client that scans for devices,}
\PYG{l+s+sd}{connects to a device (GATT server) of choice and continuously reads a characteristic on that device.}

\PYG{l+s+sd}{The GATT Server in this example runs on an ESP32 with Arduino. For the}
\PYG{l+s+sd}{exact script used for this example see {}`here \PYGZlt{}https://github.com/nkolban/ESP32\PYGZus{}BLE\PYGZus{}Arduino/blob/6bad7b42a96f0aa493323ef4821a8efb0e8815f2/examples/BLE\PYGZus{}notify/BLE\PYGZus{}notify.ino/\PYGZgt{}{}`\PYGZus{}}
\PYG{l+s+sd}{\PYGZdq{}\PYGZdq{}\PYGZdq{}}

\PYG{k+kn}{from} \PYG{n+nn}{bluepy.btle} \PYG{k+kn}{import} \PYG{o}{*}
\PYG{k+kn}{from} \PYG{n+nn}{PandaBot} \PYG{k+kn}{import} \PYG{n}{PandaBotClient}\PYG{p}{,} \PYG{n}{PandaBotDevice}

\PYG{c+c1}{\PYGZsh{} The UUID of the characteristic we want to read and the name of the device \PYGZsh{} we want to read it from}
\PYG{n}{Characteristic\PYGZus{}UUID} \PYG{o}{=} \PYG{l+s+s2}{\PYGZdq{}}\PYG{l+s+s2}{beb5483e\PYGZhy{}36e1\PYGZhy{}4688\PYGZhy{}b7f5\PYGZhy{}ea07361b26a8}\PYG{l+s+s2}{\PYGZdq{}}
\PYG{n}{Device\PYGZus{}Name} \PYG{o}{=} \PYG{l+s+s2}{\PYGZdq{}}\PYG{l+s+s2}{MyESP32}\PYG{l+s+s2}{\PYGZdq{}}

\PYG{c+c1}{\PYGZsh{} Define our scan and notification callback methods}
\PYG{k}{def} \PYG{n+nf}{myScanCallback}\PYG{p}{(}\PYG{n}{client}\PYG{p}{,} \PYG{n}{device}\PYG{p}{,} \PYG{n}{isNewDevice}\PYG{p}{,} \PYG{n}{isNewData}\PYG{p}{)}\PYG{p}{:}
    \PYG{n}{client}\PYG{o}{.}\PYG{n}{\PYGZus{}yes} \PYG{o}{=} \PYG{n+nb+bp}{True}
    \PYG{k}{print}\PYG{p}{(}\PYG{l+s+s2}{\PYGZdq{}}\PYG{l+s+s2}{\PYGZsh{}MAC: }\PYG{l+s+s2}{\PYGZdq{}} \PYG{o}{+} \PYG{n}{device}\PYG{o}{.}\PYG{n}{addr} \PYG{o}{+} \PYG{l+s+s2}{\PYGZdq{}}\PYG{l+s+s2}{ \PYGZsh{}isNewDevice: }\PYG{l+s+s2}{\PYGZdq{}} \PYG{o}{+}
          \PYG{n+nb}{str}\PYG{p}{(}\PYG{n}{isNewDevice}\PYG{p}{)} \PYG{o}{+} \PYG{l+s+s2}{\PYGZdq{}}\PYG{l+s+s2}{ \PYGZsh{}isNewData: }\PYG{l+s+s2}{\PYGZdq{}} \PYG{o}{+} \PYG{n+nb}{str}\PYG{p}{(}\PYG{n}{isNewData}\PYG{p}{)}\PYG{p}{)}
\PYG{c+c1}{\PYGZsh{} TODO: NOTIFICATIONS ARE NOT SUPPORTED YET}
\PYG{c+c1}{\PYGZsh{} def myNotificationCallback(client, characteristic, data):}
\PYG{c+c1}{\PYGZsh{}     print(\PYGZdq{}Notification received!\PYGZdq{})}
\PYG{c+c1}{\PYGZsh{}     print(\PYGZdq{}  Characteristic UUID: \PYGZdq{} + characteristic.uuid)}
\PYG{c+c1}{\PYGZsh{}     print(\PYGZdq{}  Data: \PYGZdq{} + str(data))}

\PYG{c+c1}{\PYGZsh{} Instantiate a PandaBotClient and set it\PYGZsq{}s scan callback}
\PYG{n}{bleClient} \PYG{o}{=} \PYG{n}{PandaBotClient}\PYG{p}{(}\PYG{p}{)}
\PYG{n}{bleClient}\PYG{o}{.}\PYG{n}{setScanCallback}\PYG{p}{(}\PYG{n}{myScanCallback}\PYG{p}{)}
\PYG{c+c1}{\PYGZsh{} TODO: NOTIFICATIONS ARE NOT SUPPORTED YET}
\PYG{c+c1}{\PYGZsh{} bleClient.setNotificationCallback(myNotificationCollback)}

\PYG{c+c1}{\PYGZsh{} Error handling to detect Keyboard interrupt (Ctrl+C)}
\PYG{c+c1}{\PYGZsh{} Loop to ensure we can survive connection drops}
\PYG{k}{while}\PYG{p}{(}\PYG{o+ow}{not} \PYG{n}{bleClient}\PYG{o}{.}\PYG{n}{isConnected}\PYG{p}{(}\PYG{p}{)}\PYG{p}{)}\PYG{p}{:}
    \PYG{k}{try}\PYG{p}{:}
        \PYG{c+c1}{\PYGZsh{} Search for 2 seconds and return a device of interest if found.}
        \PYG{c+c1}{\PYGZsh{} Internally this makes a call to bleClient.scan(timeout), thus}
        \PYG{c+c1}{\PYGZsh{} triggering the scan callback method when nearby devices are detected}
        \PYG{n}{device} \PYG{o}{=} \PYG{n}{bleClient}\PYG{o}{.}\PYG{n}{searchDevice}\PYG{p}{(}\PYG{n}{name}\PYG{o}{=}\PYG{l+s+s2}{\PYGZdq{}}\PYG{l+s+s2}{MyESP32}\PYG{l+s+s2}{\PYGZdq{}}\PYG{p}{,} \PYG{n}{timeout}\PYG{o}{=}\PYG{l+m+mi}{2}\PYG{p}{)}
        \PYG{k}{if}\PYG{p}{(}\PYG{n}{device} \PYG{o+ow}{is} \PYG{o+ow}{not} \PYG{n+nb+bp}{None}\PYG{p}{)}\PYG{p}{:}
            \PYG{c+c1}{\PYGZsh{} If the device was found print out it\PYGZsq{}s info}
            \PYG{k}{print}\PYG{p}{(}\PYG{l+s+s2}{\PYGZdq{}}\PYG{l+s+s2}{Found device!!}\PYG{l+s+s2}{\PYGZdq{}}\PYG{p}{)}
            \PYG{n}{device}\PYG{o}{.}\PYG{n}{printInfo}\PYG{p}{(}\PYG{p}{)}

            \PYG{c+c1}{\PYGZsh{} Proceed to connect to the device}
            \PYG{k}{print}\PYG{p}{(}\PYG{l+s+s2}{\PYGZdq{}}\PYG{l+s+s2}{Proceeding to connect....}\PYG{l+s+s2}{\PYGZdq{}}\PYG{p}{)}
            \PYG{k}{if}\PYG{p}{(}\PYG{n}{bleClient}\PYG{o}{.}\PYG{n}{connect}\PYG{p}{(}\PYG{n}{device}\PYG{p}{)}\PYG{p}{)}\PYG{p}{:}

                \PYG{c+c1}{\PYGZsh{} Have a peek at the services provided by the device}
                \PYG{n}{services} \PYG{o}{=} \PYG{n}{device}\PYG{o}{.}\PYG{n}{getServices}\PYG{p}{(}\PYG{p}{)}
                \PYG{k}{for} \PYG{n}{service} \PYG{o+ow}{in} \PYG{n}{services}\PYG{p}{:}
                    \PYG{k}{print}\PYG{p}{(}\PYG{l+s+s2}{\PYGZdq{}}\PYG{l+s+s2}{Service [}\PYG{l+s+s2}{\PYGZdq{}}\PYG{o}{+}\PYG{n+nb}{str}\PYG{p}{(}\PYG{n}{service}\PYG{o}{.}\PYG{n}{uuid}\PYG{p}{)}\PYG{o}{+}\PYG{l+s+s2}{\PYGZdq{}}\PYG{l+s+s2}{]}\PYG{l+s+s2}{\PYGZdq{}}\PYG{p}{)}

                \PYG{c+c1}{\PYGZsh{} Check to see if the device provides a characteristic with the}
                \PYG{c+c1}{\PYGZsh{} desired UUID}
                \PYG{n}{counter} \PYG{o}{=} \PYG{n}{bleClient}\PYG{o}{.}\PYG{n}{getCharacteristics}\PYG{p}{(}
                    \PYG{n}{uuids}\PYG{o}{=}\PYG{p}{[}\PYG{n}{Characteristic\PYGZus{}UUID}\PYG{p}{]}\PYG{p}{)}\PYG{p}{[}\PYG{l+m+mi}{0}\PYG{p}{]}
                \PYG{k}{if}\PYG{p}{(}\PYG{n}{counter}\PYG{p}{)}\PYG{p}{:}
                    \PYG{c+c1}{\PYGZsh{} If it does, then we proceed to read its value every second}
                    \PYG{k}{while}\PYG{p}{(}\PYG{n+nb+bp}{True}\PYG{p}{)}\PYG{p}{:}
                        \PYG{c+c1}{\PYGZsh{} Error handling ensures that we can survive from}
                        \PYG{c+c1}{\PYGZsh{} potential connection drops}
                        \PYG{k}{try}\PYG{p}{:}
                            \PYG{c+c1}{\PYGZsh{} Read the data as bytes and convert to string}
                            \PYG{n}{data\PYGZus{}bytes} \PYG{o}{=} \PYG{n}{bleClient}\PYG{o}{.}\PYG{n}{readCharacteristic}\PYG{p}{(}
                                \PYG{n}{counter}\PYG{p}{)}
                            \PYG{n}{data\PYGZus{}str} \PYG{o}{=} \PYG{l+s+s2}{\PYGZdq{}}\PYG{l+s+s2}{\PYGZdq{}}\PYG{o}{.}\PYG{n}{join}\PYG{p}{(}\PYG{n+nb}{map}\PYG{p}{(}\PYG{n+nb}{chr}\PYG{p}{,} \PYG{n}{data\PYGZus{}bytes}\PYG{p}{)}\PYG{p}{)}

                            \PYG{c+c1}{\PYGZsh{} Now print the data and wait for a second}
                            \PYG{k}{print}\PYG{p}{(}\PYG{l+s+s2}{\PYGZdq{}}\PYG{l+s+s2}{Data: }\PYG{l+s+s2}{\PYGZdq{}} \PYG{o}{+} \PYG{n}{data\PYGZus{}str}\PYG{p}{)}
                            \PYG{n}{time}\PYG{o}{.}\PYG{n}{sleep}\PYG{p}{(}\PYG{l+m+mf}{1.0}\PYG{p}{)}
                        \PYG{k}{except} \PYG{n}{BTLEException} \PYG{k}{as} \PYG{n}{e}\PYG{p}{:}
                            \PYG{c+c1}{\PYGZsh{} If we get disconnected from the device, keep}
                            \PYG{c+c1}{\PYGZsh{} looping until we have reconnected}
                            \PYG{k}{if}\PYG{p}{(}\PYG{n}{e}\PYG{o}{.}\PYG{n}{code} \PYG{o}{==} \PYG{n}{BTLEException}\PYG{o}{.}\PYG{n}{DISCONNECTED}\PYG{p}{)}\PYG{p}{:}
                                \PYG{n}{bleClient}\PYG{o}{.}\PYG{n}{disconnect}\PYG{p}{(}\PYG{p}{)}
                                \PYG{k}{print}\PYG{p}{(}
                                    \PYG{l+s+s2}{\PYGZdq{}}\PYG{l+s+s2}{Connection to BLE device has been lost!}\PYG{l+s+s2}{\PYGZdq{}}\PYG{p}{)}
                                \PYG{k}{break}
                                \PYG{c+c1}{\PYGZsh{} while(not bleClient.isConnected()):}
                                \PYG{c+c1}{\PYGZsh{}     bleClient.connect(device)}

            \PYG{k}{else}\PYG{p}{:}
                \PYG{k}{print}\PYG{p}{(}\PYG{l+s+s2}{\PYGZdq{}}\PYG{l+s+s2}{Could not connect to device! Retrying in 3 sec...}\PYG{l+s+s2}{\PYGZdq{}}\PYG{p}{)}
                \PYG{n}{time}\PYG{o}{.}\PYG{n}{sleep}\PYG{p}{(}\PYG{l+m+mf}{3.0}\PYG{p}{)}
        \PYG{k}{else}\PYG{p}{:}
            \PYG{k}{print}\PYG{p}{(}\PYG{l+s+s2}{\PYGZdq{}}\PYG{l+s+s2}{Device not found! Retrying in 3 sec...}\PYG{l+s+s2}{\PYGZdq{}}\PYG{p}{)}
            \PYG{n}{time}\PYG{o}{.}\PYG{n}{sleep}\PYG{p}{(}\PYG{l+m+mf}{3.0}\PYG{p}{)}
    \PYG{k}{except} \PYG{n}{BTLEException} \PYG{k}{as} \PYG{n}{e}\PYG{p}{:}
        \PYG{c+c1}{\PYGZsh{} If we get disconnected from the device, keep}
        \PYG{c+c1}{\PYGZsh{} looping until we have reconnected}
        \PYG{k}{if}\PYG{p}{(}\PYG{n}{e}\PYG{o}{.}\PYG{n}{code} \PYG{o}{==} \PYG{n}{BTLEException}\PYG{o}{.}\PYG{n}{DISCONNECTED}\PYG{p}{)}\PYG{p}{:}
            \PYG{n}{bleClient}\PYG{o}{.}\PYG{n}{disconnect}\PYG{p}{(}\PYG{p}{)}
            \PYG{k}{print}\PYG{p}{(}
                \PYG{l+s+s2}{\PYGZdq{}}\PYG{l+s+s2}{Connection to BLE device has been lost!}\PYG{l+s+s2}{\PYGZdq{}}\PYG{p}{)}
            \PYG{k}{break}
    \PYG{k}{except} \PYG{n+ne}{KeyboardInterrupt} \PYG{k}{as} \PYG{n}{e}\PYG{p}{:}
        \PYG{c+c1}{\PYGZsh{} Detect keyboard interrupt and close down}
        \PYG{c+c1}{\PYGZsh{} bleClient gracefully}
        \PYG{n}{bleClient}\PYG{o}{.}\PYG{n}{disconnect}\PYG{p}{(}\PYG{p}{)}
        \PYG{k}{raise} \PYG{n}{e}
\end{sphinxVerbatim}


\chapter{Indices and tables}
\label{\detokenize{index:indices-and-tables}}\begin{itemize}
\item {} 
\DUrole{xref,std,std-ref}{genindex}

\item {} 
\DUrole{xref,std,std-ref}{modindex}

\item {} 
\DUrole{xref,std,std-ref}{search}

\end{itemize}



\renewcommand{\indexname}{Index}
\printindex
\end{document}